\documentclass{article}

\textwidth 6in%

\textheight 7.9in%

\oddsidemargin 1cm



\usepackage{amsmath}

\usepackage{amsfonts}

\usepackage{amssymb}

\usepackage{graphicx}%

\usepackage[dvips]{color}

\setcounter{MaxMatrixCols}{30}

\usepackage{natbib}



\begin{document}



\title{Authors' Response \\
\begin{Large}
	Data-Driven Model Predictive Control using Random Forests for Building Energy Optimization and Climate Control
\end{Large}
}



\author{Francesco Smarra, Achin Jain, Tullio de Rubeis, \\ Dario Ambrosini, Alessandro D’Innocenzo, Rahul Mangharam}



\date{}

\maketitle
\bigskip



The authors would like to thank the Editors, the Associate Editor and the Reviewers for their constructive and helpful comments, which helped us to improve both the quality and the presentation of the paper. We performed a major revision taking into account all their comments (changes in the manuscript are highlighted in \textcolor[rgb]{0.00,0.00,1.00}{blue colored fonts}), and in particular:



\begin{itemize}



	\item Section 1 - INTRODUCTION

	\begin{itemize}

		\item The introduction has been completely re-organized and rewritten in most of its parts, with a more precise and detailed explanation of the paper contribution;

		\item Literature review and related work have been updated and extended;

	\end{itemize}



	\item Section 2 - COMPLEXITY ISSUES IN MODELING BUILDINGS: PHYSICS-BASED VS DATA-DRIVEN

	\begin{itemize}

		\item This section has been added to the paper to highlight in detail the advantages of data-based modeling with respect to physics-based modeling for buildings;

	\end{itemize}



	\item Section 3 - DATA PREDICTIVE CONTROL

	\begin{itemize}

		\item We rewrote the section to make the presentation more clear;

		\item We provided additional details to make the mathematical  description more clear and rigorous;

	\end{itemize}



	\item Section 6 - APPLICATION TO OPTIMAL HEATING SYSTEM SCHEDULING

	\begin{itemize}

		\item We highlighted the differences in contribution with respect to Section 5;

		\item We added a new section with new results that account for weather forecast uncertainties and show the robustness of the proposed approach;

	\end{itemize}



	\item Other changes

	\begin{itemize}

		\item We updated the abstract to make it more complete and clear;

		\item We performed a deep review to improve both language and organisation quality;

		\item We improved the notation and corrected some typos;
		\item We added Appendix A to provide further technical details describing the classical Regression Tree algorithm.


	\end{itemize}



\end{itemize}



The following sections consist of itemized lists of the detailed authors' response to the Reviewers, to the Editors and to the Associate Editor.


%================================================================================


\section{Response to the Associate Editor}



\begin{enumerate}

	\item \textbf{Associate Editor:} \textit{The authors should address the points and clarifications in the reviews, and also specifically address the novelty of the paper with respect to prior work.}

	

	\textbf{Authors:} We addressed all the reviewers' points, as discussed in the following of this document. We also emphasized the novelty of the paper with respect to prior work in the "Related work" subsection of the introduction.
	
\end{enumerate}
	
\section{Response to the Editors}



\begin{enumerate}

	

	\item \textbf{Editor:} \textit{An updated and complete literature review should be conducted.}

		

	\textbf{Authors:} We conducted and updated and complete literature review. In particular, in Table 1 we classify the related work and then emphasize the novelty of our research with respect to each prior work.



	\item \textbf{Editor:} \textit{The relevance to Applied Energy should be enhanced with the considerations of scope and readership of the Journal. The present form has little relevance to APEN which might suit better for another journal. Authors must address and enhance the relevance to Applied Energy otherwise, they can consider to publish in another journal.}

	

	\textbf{Authors:} In the initial submission we had chosen a more general title and discussion in the introduction because we believe that our approach has the potential to be applied not only to building management, but also to other application domains. However, as in this paper we develop our techniques focusing and experimenting in building management we changed the title in "Data-Driven Model Predictive Control using Random Forests for Building Energy Optimization and Climate Control" and re-wrote the introduction of the paper focusing on the relevance of the paper contribution to building management.\\
	Now that we better emphasised the relevance of the paper to building management, and since "\textit{Applied Energy provides a forum for information on innovation, research, development and demonstration in the areas of energy conversion and conservation, the optimal use of energy resources, analysis and optimization of energy processes, mitigation of environmental pollutants, and sustainable energy systems" (https://www.journals.elsevier.com/applied-energy)}, we really hope that our paper will be considered relevant to APEN, especially with respect to the topics "\textit{optimal use of energy resources}" and "\textit{analysis and optimization of energy processes}". We also hope that our paper will be considered related to the APEN keywords \textit{Energy efficiency and energy saving: strategies, applications and implications}, \textit{Energy system analysis, simulation and modeling}, \textit{Implementation of energy R\&D strategies and policies}, \textit{Energy system and automation} and \textit{Sustainable buildings}.


	\item \textbf{Editor:} \textit{A proof reading by a native English speaker should be conducted to improve both language and organization quality.}



	\textbf{Authors:} We deeply revised the paper to improve both language and organization quality.



	\item \textbf{Editor:} \textit{The originality of the paper needs to be further clarified. The present form does not have sufficient results to justify the novelty of a high quality journal paper. The results should be further elaborated to show how they could be used for the real applications.}



	\textbf{Authors:} In the "Related work" and "Main contribution" subsections of the Introduction, we clarified the novelty of the paper with respect to existing results and emphasized the contributions provided.\\
		In the newly added Section 2 we highlight the typical drawbacks of the model-based approaches, which limit the use of advanced control techniques for energy optimization such as MPC due to the difficulties (cost, time and effort) related to the model creation, and show how the data-driven approach that we propose (DPC) overcomes these limitation.\\
		In Sections 4,5 and 6 we show how our DPC methodology can be used for real applications. In particular, in Section 4 we demonstrate that DPC can provide comparable performance with respect to classical MPC applied to a perfectly known mathematical model. In Section 5 we apply DPC to a 6 story 22 zone building model in EnergyPlus, for which model-based control is not economical and practical due to extreme complexity, and address a Demand Response problem. Our results demonstrate scalability and efficiency of DPC showing that it provides the desired power curtailment with an average error of 3\%. In Section 6 we implement and test DPC on real data from an off-grid house located in L'Aquila, Italy. We compare the total amount of energy saved with respect to the classical bang-bang controller, showing that DPC can perform an energy saving up to $49.2\%$. We also demonstrate the robustness of our method to uncertainties both in real data acquisition (affected by unpredictable house schedules due to people, appliances, etc., noisy sensor measurements due to real experimental data, etc.) and weather forecast.
\end{enumerate}



%================================================================================



\section{Response to Reviewer 1}



\begin{enumerate}

\item \textbf{Reviewer 1:} \textit{This paper presents a new framework called data predictive control which integrates machine learning and model predictive control to derive the optimal online control decisions. The framework is then applied to study several demand response problems in energy area as case studies. The problem considered in this paper is important, and the proposed approach is a good initial attempt to address the problem relying on heuristics.}



\textbf{Authors:} We thank the reviewer for the comments on the problem and approach of our paper.



\item \textbf{Reviewer 1:} \textit{In Section 2.1, what is the relationship between these training features and labels $(X^c, X^d, Y)$ and the input, output, disturbance, and state variables in the traditional state space model? Are they the same?}



\textbf{Authors:} We thank the reviewer for this comment, this very important part of the paper was not clear in the initial submission. We added a new section (Section 2) where we reported a traditional physics-based state-space modeling of a building (Section 2.1) and compared it with the data-driven modeling approach (Section 2.2). We highlighted the differences in the modeling techniques and emphasized that the features needed in the data-driven approaches are ALL and ONLY the measurable variables (hence a subset of all the variables) used in the model-based approach, since the model learning procedure is able to compensate the effect of the non-measurable variables. This is also further discussed in Section 3, where we added more mathematical details to make the presentation more clear and rigorous.



\item \textbf{Reviewer 1:} \textit{Another point is that in traditional state space model, one cannot measure everything except input and output in most cases. How could we obtain such training data if these data are not measurable.}



\textbf{Authors:} We thank the reviewer for this comment, indeed our previous presentation of the paper was unclear on this fundamental aspect, we apologise for this. In our revision we widely illustrate in Section 2, how in traditional state-space models part of the state is measurable (i.e. room temperatures, etc) and part is not measurable (i.e. layers/windows temperatures, etc). In our approach, of course, $(\mathcal{X}, \mathcal{Y})$ are ALL measurable variables, i.e. variables that are present in the historical data of a building. The unmeasurable variables of the traditional state-space models are not assumed to be measurable in this paper, yet our prediction works well since we compensate their effect via the regression tree/random forest approaches.



\item \textbf{Reviewer 1:} \textit{The key intuition of the proposed framework is Eq. (1). It would be much better if the authors can explain their framework using Eq. (1) directly first instead of using another set of notations such as $(X^c, X^d, Y)$. The relationship between Eq. (1) and Fig. (1) is not very clear. Specifically, one would expect to estimate $g$ and $h$ using data, but how they can be trained in a two-step process as shown in Fig. 1 is not clear.}



\textbf{Authors:} We agree with the reviewer. Following this comment, and also other comments, we re-wrote the corresponding Section 3 (describing the DPC technical approach) to make the presentation and notation more clear and the mathematical description more rigorous. We first replaced Eq.(1) of the initial submission with Eq.(5), i.e. $y(k)=f(x(k),u(k),d(k))$, of the current version. In Section 3.1 we re-wrote in a more clear and detailed way the two-steps process (based on an adaptation of the Regression Tree algorithm) used to construct from the historical data the predictive model $f$, and we also clarified the relationship between such two-steps process and Fig. 1 (left). We also added a remark that better explains Fig. 1 (right) and Appendix A to provide further technical details describing the classical Regression Tree algorithm.


\end{enumerate}


%================================================================================



\section{Response to Reviewer 2}



\begin{enumerate}

\item \textbf{Reviewer 2:} \textit{The paper applies a recently developed concept, data predictive control (DPC) to building energy control. Models based on decision trees and random forests are applied. The efficacy of the approach appears well validated.}



\textbf{Authors:} We thank the Reviewer for the comment on our validation of the efficacy of our approach.



\item \textbf{Reviewer 2:} \textit{The title is a bit confusing as it's very general. It should be changed to be more specifically focusing on the applications in this paper.}



\textbf{Authors:} We agree with the Reviewer. In the initial submission we had chosen a more general title (and discussion in the introduction) because we believe that our approach has the potential to be applied not only to building management, but also to other application domains. However, as in this paper we develop our techniques focusing and experimenting in building management (which is indeed a core topic of APEN) we changed the title in "Data-Driven Model Predictive Control using Random Forests for Building Energy Optimization and Climate Control" (and re-wrote the introduction of the paper focusing on the relevance of the paper contribution to building management). 



\item \textbf{Reviewer 2:} \textit{Also, novelties beyond what's in [9] - [12] should be clarified.}



\textbf{Authors:} In subsection "1.1 Related work" of the revised version we discuss in detail the novelty of our paper beyond [9,10,11,12] (which are numbered [25,26,31,32] in the revised version). We summarise the novel contributions of our paper in the initial paragraph of subsection "1.2 Main contribution":
\textit{In this paper, we provide a new methodology based on random forests that overcomes the drawbacks of all our previous works, and more precisely: we obtain better performance and scalability when compared to other approaches (both optimal and rule-based), and we provide and validate robustness with respect to uncertainties due to real data acquisition and weather forecast inaccuracies.}
After the above paragraph in the paper, we describe our contributions in detail.



\item \textbf{Reviewer 2:} \textit{In (1) and 2.1, the description of separation of variables is vague and needs to be more precise. E.g., the line below (1) does not make sense. It is not until later seeing the examples the meaning of this part becomes clearer. Such confusion persists up to and beyond (3). Mathematically, (2) and (3) cannot be simultaneously true. While I can see the authors trying to be descriptive, the descriptions lack rigor.} 



\textbf{Authors:} We agree with the reviewer. Following this comment, and also other comments, we re-wrote the corresponding Section 3 (describing the DPC technical approach) to make the presentation and notation more clear and the mathematical description more rigorous. We also added Appendix A to provide further technical details describing the classical Regression Tree algorithm.


\item \textbf{Reviewer 2:} \textit{The main optimization problems (4) and (7) are not clear in themselves. As you look ahead to do DPC, does each future time slot have a different linear model, depending on future non-manipulative states? Then, do you assume you know precisely all the future non-manipulative states? If so, can you justify this? Apparently, with a little error in predicting of the future states, a completely different linear model may be chosen. Will this change the performance significantly? Even with forests this is a major concern.}



\textbf{Authors:} We thank the reviewer for rising this important issue. As discussed in the previous answer, in this revision we re-wrote Section 3 and improved the description of the main optimization problems. Answering to the specific questions of the reviewer: yes, as we look ahead to do DPC each future time slot have a different linear model, depending on future non-manipulative states; yes, we assume that at time $k$ we know the future non-manipulative states (namely external temperature, radiation, etc.) for the Model Predictive Control time horizon. In the revised version of this paper, to justify this assumption, we first provide a reference paper [57] that investigates the effect of uncertainty in the weather forecast on the performance of MPC in building systems operations through a large-scale simulation study, and compares against a rule-based strategy. They consider 48 different scenarios of uncertainties for 72 hours weather forecast. With such a long horizon, results have shown that (with a few exceptions) MPC outperforms the rule-based controller in providing energy savings, and is in general quite close to the perfect forecast case despite the uncertainty in weather forecast. Since the length of horizon for the DPC algorithm presented in this paper is usually much shorter, for example $6$ hours in Section 4, $7$ hours in Section 5, and $40$ minutes in Section 6, we can reasonably presume that our approach will be robust to weather forecast inaccuracies. In our opinion, robustness of DPC to weather forecast inaccuracies is due to 2 main features:

	\begin{itemize}

	\item weather forecast are in general quite accurate on a short-term horizon;

	\item the MPC strategy computes an optimal input sequence over an horizon of finite length, and then applies only the first input. At the next step, with the new measurements of the system state and of the disturbance conditions, the algorithm is applied again. For this reason MPC is in general very robust to uncertainties;

	\end{itemize}
	Finally, in Section 6 we add a new subsection where we test the robustness of our approach by perturbing the weather forecast with Gaussian noise with large variance. The results show that the control performance is indeed very close to the ideal case even with a large error in the weather forecast.



\item \textbf{Reviewer 2:} \textit{In Fig. 6b, what's the difference between blue and yellow, as they seem to completely overlap?}



\textbf{Authors:} We have clarified this point in the text and the Figure caption. In particular, marked in blue is the response when the optimal input is applied to the power predictive model of the Random Forest and in red is the response when the same input is applied to the power predictive model of EnergyPlus. Since the optimal input is computed using the power predictive model of the Random Forest the blue trajectory perfectly follows the tracking signal. The red trajectory, as expected, is characterised by a (small) tracking error because of the model mismatch between the predictive model of the Random Forest (used to compute the input) and that of EnergyPlus (used to simulate the closed-loop system).



\item \textbf{Reviewer 2:} \textit{Also, what is new in section 5 compared with section 4? Even though a real house is claimed to be used, it's still tested using simulations and energyplus, like in the previous sections. A real world experiment on this house would be much more interesting.}



\textbf{Authors:} We thank the reviewer for this comment, indeed the contribution of Section 6 (i.e. Section 5 in the initial submission) was not well emphasized. We now emphasize the difference between Section 5 (i.e. Section 4 in the initial submission) and Section 6 (i.e. Section 5 in the initial submission):
\begin{itemize}

	\item As the main difference, in Section 5 we use the EnergyPlus model both to create the predictive model via Random Forests and to test the DPC performance. However, although the EnergyPlus model is quite accurate, it still provides simulated data, that differently from experimental data do not suffer of non-predictable imperfections such as random occupancy schedules, open/close windows, random light on/off switch, etc.. In Section 6, instead, real data are used to build the Random Forests models, while the EnergyPlus model (created ad-hoc for the house) is ONLY used to compare the total amount of energy can be saved applying DPC with respect to the classical bang-bang controller.

	\item The optimization problems are different: in Section 5 we cut the peak of power consumption in a limited amount of time and require the temperature to be close to a desired value, while in Section 6 we minimize the energy usage while keeping temperature in a comfort range. Furthermore, in Section 6 we consider an on/off input, i.e. an integer input variable, instead of a continuous one, showing that  DPC can easily adapt to different environments in practice;
	\item Last but not least, we add in the revised version a new subsection of Section 6 adding simulations accounting for weather forecast uncertainties, other than the ones introduced with the experimental data, that show the robustness of DPC against weather forecast inaccuracies.

\end{itemize}
Finally, we agree with the reviewer that a real world experiment would be interesting. However, the aim of Section 6 was to compare the impact in energy saving of DPC and of the classical bang-bang controller: since they could not be ran experimentally simultaneously our simulative approach appears reasonable. Now that we have demonstrated in this paper \textit{performance}, \textit{scalability} and \textit{robustness} of DPC we are currently working to apply it on a real building of the University of L'Aquila (Italy), but this study is out of the scope of this paper.

\end{enumerate}



\end{document}



