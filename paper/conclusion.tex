
We present two algorithms based on trees and random forests for receding horizon control with data-driven models. 
We compare the performance of our Data Predictive Control to MPC on a multivariable bilinear building model. We establish that DPC with random forests shows a remarkable similarity to MPC in the optimal control strategies explaining 70\% variance. On the other hand, DPC with regression trees suffers from practical limitations due to model overfitting.
We further apply DPC with random forests to a large scale 6 story EnergyPlus model with 22 zones for which the traditional model-based control is largely unsuitable due to complex dynamics and the cost of model identification. We show that DPC, relying only on the sensor data, can provide significant energy savings while maintaining thermal comfort. Our results demonstrate that even for such complex system, DPC tracks a reference signal with a mean error of 3\%.

DPC has applications which go beyond buildings and energy systems, to industrial process control, and controlling large critical infrastructures like water networks, district heating \& cooling. DPC is immensely valuable in situations where first principles based modeling cost is extremely high.