\textcolor[rgb]{0,0,1}{Model Predictive Control (MPC) is a model-based technique widely and successfully used over the past years to improve control systems performance. A key factor prohibiting the widespread adoption of MPC for complex systems such as buildings is related to the difficulties (cost, time and effort) associated with the identification of a predictive model of a building.\\
To overcome this problem, we introduce a novel idea for predictive control based on historical building data leveraging machine learning algorithms like regression trees and random forests. We call this approach Data-driven model Predictive Control (DPC), and we apply it to three different case studies to demonstrate its \textit{performance}, \textit{scalability} and \textit{robustness}.\\
In the first case study we consider a benchmark MPC controller using a bilinear building model, then we apply DPC to a data-set simulated from such bilinear model and derive a controller based only on the data. Our results demonstrate that DPC can provide comparable performance with respect to MPC applied to a perfectly known mathematical model.\\
In the second case study we apply DPC to a 6 story 22 zone building model in EnergyPlus, for which model-based control is not economical and practical due to extreme complexity, and address a Demand Response problem. Our results demonstrate scalability and efficiency of DPC showing that DPC provides the desired power curtailment with an average error of 3\%.\\
In the third case study  we implement and test DPC on real data from an off-grid house located in L'Aquila, Italy.
%We design the optimal ON/OFF scheduling for the heating system in order to save energy while guaranteeing thermal comfort for the occupants.
We compare the total amount of energy saved with respect to the classical bang-bang controller, showing that we can perform an energy saving up
%that ranges from $25.4\%$ (if we guarantee thermal comfort i.e.~strictly respect the desired temperature range in the rooms)
to $49.2\%$.
%(if we allow small violation in the desired temperature range).
Our results demonstrate robustness of our method to uncertainties both in real data acquisition and weather forecast.}

