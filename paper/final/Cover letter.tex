\documentclass[12pt]{article}

\begin{document}

\title{Cover Letter}
\date{}
\author{}
\maketitle 
\thispagestyle{empty}

In this paper we introduce a novel idea for predictive control based on historical building data, leveraging machine learning algorithms like regression trees and random forests. We call this approach Data-driven model Predictive Control (DPC), and we apply it to three different case studies to demonstrate its performance, scalability and robustness. This is the first time that Random Forests are bridged towards Predictive Control theory, and used to optimize energy consumption, thermal comfort, and peak power reduction in buildings. This approach overcomes the complexity drawbacks of the classical model-based approaches, that limit their widespread in real applications, and provides better results with respect to the current used rule-based approaches, while being extremely simple to be implemented in C, C++, Python and Matlab, and thus is easily integrable in SCADA systems.

\paragraph{What is the novelty of this work?}
We introduce a novel idea for predictive control based on historical building data leveraging machine learning algorithms like regression trees and random forests. We call this approach Data-driven model Predictive Control (DPC). This is the first time Random Forests based algorithms are bridged with the Predictive Control theory. This approach has shown its performance, scalability and robustness, with respect to uncertainties due to real data acquisition and weather forecast inaccuracies, and under different scenarios of energy and thermal comfort optimization problems, i.e. Demand Response and peak power curtailment, energy and thermal comfort optimization, in buildings.

\paragraph{Is the paper appealing to a popular or scientific audience?}
This paper is appealing for a vast audience (taken from the list available on the Applied Energy's guideline):
\begin{itemize}
	\item Mechanical, and alternative renewable energy sources including biomass: since, (1) a real house has been considered making a study on the thermal properties of the structure, and (2) a thermal plant (based on biomass) schedule has been optimized for energy efficiency;
	\item Building, policy and decision makers in energy conservation and conversion, and sustainable energy systems : since the whole paper focuses on a new methodology, that is data-driven, for energy and thermal comfort optimization, and peak power reduction in buildings;
	\item Researchers: in the paper we create a new methodology to bridge machine learning and control, so the work is appealing also for these two communities.
	
\end{itemize}

\paragraph{Why the authors think the paper is important and why the journal should publish it?}
The paper introduce a novel methodology for energy and thermal comfort optimization in buildings. In particular, this methodology provides a new data-driven control technique, Data-driven model Predictive Control (DPC), that offer many advantages in terms of practical applicability with respect to the classical model-based approaches, and improves the performance of the rule-based approaches, while being extremely simple to be implemented in real applications.
\paragraph{Has the article been checked by a native tongue speaker with expertise in the field?}
Yes
\paragraph{Are you available as a reviewer for at least three other articles for Applied Energy during the current year?}
Yes

\end{document}


