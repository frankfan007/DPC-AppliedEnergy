\section{INTRODUCTION}
\label{S:intro}

Machine learning and control theory are two foundational but disjoint communities. Machine learning requires data to produce models, and control systems require models to provide stability and performance guarantees to plant operations. Machine learning is widely used for regression or classification, but thus far data-driven models have not been suitable for closed-loop control of physical plants. The challenge now, with using data-driven approaches, is to close the loop for real-time control and decision making.

\subsection{Motivation}
TO IMPROVE $\rightarrow$ \textcolor[rgb]{0,0,1}{Achin and Francesco}

%Consider a multivariable dynamical system subject to external disturbances. The first and foremost requirement for making any decision is to obtain the underlying control-oriented predictive model of the system. With a reasonable forecast of the external disturbances, these models should predict the state of the system in the future and thus Model Predictive Control (MPC) can act preemptively to provide a desired system behavior while optimizing a desired performance. In particular, MPC has been proven to be very powerful for multivariable systems in the presence of input and output constraints, and forecast of the disturbances. The caveat is that MPC requires a reasonably accurate physical representation of the system. This makes MPC unsuitable for control of complex plants such as natural gas processing, oil refineries, boilers, manufacturing plants, and buildings where the user expertise, time, and associated sensor costs required to develop a model are very high \cite{Sturzenegger2016,vzavcekova2014}.
%
%There are two main reasons for model complexity. 
%(1) The prime contributor is the change in model properties over time. Even if the model is identified once via an expensive route, as the model changes with time, the system identification must be repeated to update the model. Thus, model adaptability or adaptive control is desirable for such systems. 
%(2) A secondary reason is the model heterogeneity which further prohibits the use of model-based control. For example, unlike the automobile or the aircraft industry, each building is designed and used in a different way. Therefore, this modeling process must be repeated for every new building. 
%Due to aforementioned reasons, the control strategies in such systems are often limited to fuzzy logic rules that are based on best practices. 


%\paragraph{Example} As an example of such modeling complexity, consider the grey-box approach in \cite{Braun2002}. The scope of such approach is to predict the heat transfer rate to the air within the building. In particular, the authors addressed a single zone modeling problem using an equivalent $RC$ model. To this end, five different types of structures are considered in the example: external walls, ceiling/roof, floor, internal walls and windows. Each of these elements are represented using $3$ resistances, $2$ capacitances and $2$ temperatures, except for the windows that are represented only with resistances since they have a negligible energy storage. A model consisting of $8$ states and $9$ inputs, including disturbances such as solar radiation and others, with the instantaneous heat gain to the building air from all surfaces as output, is created for the zone. To have a complete LTI model, $9$ parameters are needed to be estimated via a non-trivial training process, using a collection of information associated with a physical description of the building (see \cite{Braun2002} for more details). Extending this result, for the building we consider in Section \ref{S:realCaseStudy} with $10$ different zones, we would need to consider a model with approximately $80$ states and $90$ parameters to be estimated.

\subsubsection{physics-based modeling example}
TO BE DONE $\rightarrow$ \textcolor[rgb]{0,0,1}{Achin and Francesco}
\begin{itemize}
	\item Introduce example
	\item add modeling equations
	\item quantify complexity
\end{itemize}

\subsubsection{data-driven modeling example}
TO BE DONE $\rightarrow$ \textcolor[rgb]{0,0,1}{Achin and Francesco}
\begin{itemize}
	\item show procedure
	\item quantify complexity
\end{itemize}

\subsubsection{physics-based vs data-driven modeling example}
TO BE DONE $\rightarrow$ \textcolor[rgb]{0,0,1}{Achin and Francesco}
\begin{itemize}
	\item compare results
\end{itemize}

\paragraph{Objective} The question now is, can we employ data-driven techniques to reduce the cost of modeling and still exploit the benefits that MPC has to offer? Can we build automatic and data-driven approaches for control purposes that are also adaptive, scalable and interpretable? In this paper we address this problem introducing \textit{Data Predictive Control (DPC)} to bridge the gap between Machine Learning and Predictive Control.

\subsection{Related work}
TO IMPROVE $\rightarrow$ \textcolor[rgb]{1,0,0}{Tullio}
%In the literature, several studies deal with the data-driven control techniques using machine learning algorithms \cite{Hou2013}. For example, Artificial Neural Networks are exploited to create data-driven models to be used as plant simulator in closed-loop with a Supervisory Model Predictive Control in \cite{Afram2017}. In \cite{Macarulla2017} the authors proposed a predictive control strategy based on Neural Networks, for boilers control in buildings, to decide the optimal time to switch-on the plan to guarantee energy saving and thermal comfort. However, the strategy is not easily scalable to different types of plants and does not use optimization theory in the closed-loop scheme. In \cite{Costanzo2016} a reinforcement learning control strategy, called Model-Assisted Batch Reinforcement Learning, is considered to provide data-driven control for the demand response problem in HVAC systems. Reinforcement Learning is a model-free methodology and for this reason it can not be used for on-line optimization in a closed-loop predictive control scheme. In \cite{Ferreira2012} the authors considered a data-driven predictive control based on Neural Networks to guarantee energy saving and  thermal comfort in public buildings. Neural Networks are used in the closed-loop control scheme to determine a thermal comfort index based on parameters that can be measured or estimated, but no  system dynamics with internal state are included into the optimization problem. More papers related to this topic can be found in the literature, but to the best of the authors' knowledge none of them address the problem of including data-driven state models in the optimal predictive control loop, and hence allow to set up a MPC-like optimal control problem, as we do in DPC.

\subsection{Main contribution and paper organization}
TO CLARIFY $\rightarrow$ \textcolor[rgb]{0,0,1}{Achin and Francesco}
\begin{itemize}
	\item Add more details
	\item Emphasize difference between Section 4 and Section 5
	\item Cite appendix
\end{itemize}
TO ADD $\rightarrow$ \textcolor[rgb]{1,0,0}{Tullio}
\begin{itemize}
	\item Add references for weather forecast accuracy
	\item Quantify accuracy for short-term prediction for weather forcast to justify the assumption of perfect knowledge for the disturbance
\end{itemize}
TO ADD $\rightarrow$ \textcolor[rgb]{0,0,1}{Achin and Francesco}
\begin{itemize}
	\item Add subsection in Section 5 with simulation of DPC considering disturbance with added noise and show that we still have good performance
\end{itemize}
%In our previous work \cite{Behl2016,Jain2016,JainACC2017,JainCDC2017}, we introduced the concept of DPC for receding horizon control. In this paper, we extend these results providing the following contributions.
%\begin{enumerate}
%	\item In Section \ref{S:dpc}, we formally present the two following control techniques:
%	\begin{enumerate}
%		\item DPC with regression trees, and
%		\item DPC with random forests.
%	\end{enumerate}
%	\item In Section \ref{S:proof}, we demonstrate the strength of DPC for receding horizon control via one-to-one comparison against a benchmark MPC controller using a bilinear building model whose parameters were identified using experiments on a building in Switzerland. We show that DPC captures 70\% variance in MPC and offers a comparable performance.
%	\item Section \ref{S:casestudy} describes a practical application of DPC for Demand Response, where we apply DPC to a 6 story 22 zone building model in EnergyPlus \cite{Crawley2001} for which model-based control is not economical and practical due to extreme complexity. We show scalability and efficiency of DPC in providing financial incentives to the end-customers bypassing the need for high fidelity models. We observe that DPC provides the desired power reduction with an average error of 3\%.
%	\item In Section \ref{S:realCaseStudy}, we implement DPC on the real data from an off-grid house located in L'Aquila (Italy), to find the optimal ON/OFF scheduling for the heating system in order to save energy while guaranteeing thermal comfort for the occupants. We quantify the total amount of energy saved, with respect to the classical bang-bang controller widely used in houses for temperature control, using an EnergyPlus model built specifically for the house. We show we can perform an energy saving that goes from $25.4\%$, if we guarantee thermal comfort, to $49.2\%$, if we allow very little discomfort in terms of rooms' temperature.
%\end{enumerate}